\chapter{Исследовательский раздел}

В данном разделе будет проведено сравнение временных характеристик обычного алгоритма z-буфера и его распаралелленной версии.

\section{Технические характеристики}

Технические характеристики устройства, на котором выполнялось тестирование и исследование, приведены ниже.

\begin{itemize}
	\item Операционная система: Ubuntu Linux 64-bit.
	\item Оперативная память: 16 GB.
	\item Количество логических ядер - 8.
	\item Процессор: Intel(R) Core(TM) i7-8850H CPU @ 2.60GHz \cite{intel}.
\end{itemize}

Тестирование проводилось на компьютере, включенном в сеть электропитания. Во время тестирования компьютер был нагружен только встроенными приложениями окружения рабочего стола, окружением рабочего стола, а также непосредственно системой тестирования. Во время тестирования оптимизации компилятора были отключены.


\section{Замеры времени}

В связи с работой алгоритма в нескольких потоках, измерялось реальное время работы, а не процессорное. Время замерялось с помощью функции now \cite{clock}.

Результаты замеров времени (в мс) приведены в таблицах \ref{tbl:tbase} - \ref{tbl:time}. На рисунке \ref{plt:time_even} приведены зависимости времени работы алгоритмов от количества поверхностей модели.

\begin{table}[H]
	\begin{center}
		\caption{Замер времени работы стандартного алгоритма для разного числа поверхностей}
		\label{tbl:tbase}
		%x*c|
		\begin{tabular}{|c|c|}
			\hline
			& \multicolumn{1}{c|}{\bfseries Время, мс}                                    \\ \cline{2-2}
			\bfseries Число поверхностей & \bfseries Стандартный алгоритм
			\csvreader{inc/tbase.csv}{}
			{\\\hline \csvcoli&\csvcolii}
			\\\hline
		\end{tabular}
	\end{center}
\end{table}

\begin{table}[H]
	\begin{center}
		\caption{Замер времени работы распараллеленного алгоритма для разного числа поверхностей}
		\label{tbl:time}
		%x*c|
		\begin{tabular}{|c|c|c|c|c|c|c|}
			\hline
			& \multicolumn{6}{c|}{\bfseries Время в мс на n потоков}                                    \\ \cline{2-7}
			\bfseries Число рёбер & \bfseries 1 & \bfseries 2 & \bfseries 4 & \bfseries 8 & \bfseries 16 & \bfseries 32
			\csvreader{inc/table.csv}{}
			{\\\hline \csvcoli&\csvcolii&\csvcoliii&\csvcoliv&\csvcolv&\csvcolvi&\csvcolvi}
			\\\hline
		\end{tabular}
	\end{center}
\end{table}

\begin{figure}[H]
	\centering
	\begin{tikzpicture}
	\begin{axis}[
	axis lines=left,
	xlabel=Количество поверхностей,
	ylabel={Время, мс},
	legend pos=north west,
	ymajorgrids=true
	]
	\addplot table[x=Len,y=th0,col sep=comma] {inc/tbase.csv};
	\addplot table[x=Len,y=th1,col sep=comma] {inc/table.csv};
	\addplot table[x=Len,y=th2,col sep=comma] {inc/table.csv};
	\addplot table[x=Len,y=th4,col sep=comma] {inc/table.csv};
	\addplot table[x=Len,y=th8,col sep=comma] {inc/table.csv};
	\addplot table[x=Len,y=th16,col sep=comma] {inc/table.csv};
	\addplot table[x=Len,y=th32,col sep=comma] {inc/table.csv};
	\legend{Стандартный, 1 поток, 2 потока, 4 потока, 8 потоков, 16 потоков, 32 потока}
	\end{axis}
	\end{tikzpicture}
	\captionsetup{justification=centering}
	\caption{Зависимость времени работы алгоритма z буфера от количества поверхностей.}
	\label{plt:time_even}
\end{figure}

\newpage
\subsection*{Вывод}

Из графиков отношения количества граней ко времени работы видно, что расчёт на одном потоке работает примерно столько же времени, как и стандартный алгоритм (разница около 1\%). При использовании двух и более потоков, время работы алгоритма уменьшается в разы. В случае, когда количество потоков равно числу логических ядер процессора, алгоритм работает в 2 (при размерности 250) и в 3 (при 1250 поверхностях) раза быстрее. При использовании количества потоков, большего, чем число логических ядер процессора, время работы изменяется в пределах 2-3\%, относительно предыдущего рассмотренного случая.
Максимальной эффективности от параллельных вычислений добиться не удалось, так как между потоками возникает конфликт по данным, для решения которого используется мьютес.