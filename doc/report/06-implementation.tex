\chapter{Технологический раздел}

В этом разделе будут рассмотрены средства разработки программного продукта и детали его реализации.

\section{Архитектура приложения}

Разрабатываемое приложение будет построено по микросервисной архитектуре. Программный продукт можно поделить на три основные части:
\begin{itemize}
	\item api\_gateway -- frontend, который принимает все запросы, строит план их исполнения и занимается балансировкой нагрузки на другие микросервисы;
	\item details -- обрабатывает все запросы, которые связаны с запчастями;
	\item workers -- работа с сотрудниками сервиса.
\end{itemize}

Обобщенная структурная схема архитектуры приложения представлена на рисунке \ref{img:architectire}.

\imgw{architectire}{H}{0.8\textwidth}{Архитектура приложения}

\section{Средства реализации}

Для реализации программного продукта был выбран язык программирования С++ \cite{cpp}. Этот язык является компилируемым, что дает преимущество в скорости в сравнении с интерпретируемыми языками программирования. С++ поддерживает различные механизмы объектно-ориентированного программирования такие как инкапсуляция, наследование и полиморфизм. Эта парадигма сочетается с задачей хранения и обработки объектов реального мира, так как каждый объект можно задать с помощью отдельного класса с соответствующим набором полей и методов \cite{oop}.

Также для С++ существует фреймворк Boost \cite{boost}, с помощью которого можно создавать асинхронные подключения, а, начиная с стандарта C++20, это делается с использованием корутин.

Для подключения к СУБД использовались фреймворки pqxx \cite{pqxx} и redis-cpp \cite{redis-cpp}.

Для создания модульных тестов был выбран фреймворк GoogleTests \cite{gtest}. Данный выбор обусловлен тем, что он включает в себя определения основных классов и макросов, которые используются для создания модульных тестов.

Для тестирования API использовался postman \cite{postman}. Он позволяет заранее заготавливать все запросы, а потом в формате одной коллекции проверять результат их исполнения.

\section{Детали реализации}

Пример взаимодействия приложения с PostgreSQL представлен на листинге \ref{lst:postgres}

\begin{lstinputlisting}[label=lst:postgres,caption=Взаимодействие приложения с PostgreSQL, language=lisp]{inc/lst/details_repository.cpp}
\end{lstinputlisting}

Пример взаимодействия приложения с Redis представлен на листинге \ref{lst:redis}

\begin{lstinputlisting}[label=lst:redis,caption=Взаимодействие приложения с Redis, language=lisp]{inc/lst/auth_repository.cpp}
\end{lstinputlisting}


\section{Взаимодействие с программой}
Запрос на получение информации о детали представлен на рисунке \ref{img:details_get}

\imgw{details_get}{H}{0.8\textwidth}{Получение информации о детали}

Запрос на авторизацию в системе представлен на рисунке \ref{img:auth_post}
\imgw{auth_post}{H}{0.8\textwidth}{Авторизация}

\section{Вывод}
В данном разделе были выбраны средства реализации программного продукта, приведены примеры взаимодействия с программой и листинги связывания приложения с различными СУБД.