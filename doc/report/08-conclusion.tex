\chapter*{ЗАКЛЮЧЕНИЕ}
\addcontentsline{toc}{chapter}{ЗАКЛЮЧЕНИЕ}

В рамках курсового проекта были решены следующие задачи:
\begin{itemize}
	\item изучены различные подходы к построению реалистичных сцен
	\item выбран наиболее подходящий под условие задачи алгоритм
	\item создана схема выбранного алгоритма
	\item определены структуры данных
	\item реализован алгоритм
	\item проведено тестирование
	\item создана версию алгоритма, которая может выполняться с использованием нескольких потоков одноверменно
	\item сравнены временные показатели обычной и распараллеленной версий
\end{itemize}

Была создана программа, способная создавать трехмерную модель планетарной системы, а также моделировать ее движение. Были реализованы автосборка и автотестирование на платформе gitlab.

В результате исследование было выяснено, что версия алгоритма с использованием параллельных вычислений может дать выигрыш во времени в 2 или 3 раза в зависимости от числа поверхностей.

Поставленная цель достигнута.